\documentclass{article}
\begin{document}
\title{RAID Storage System}


\author{Sachin Malve \\
	19111049 \\
 	6th Semester \\ 
	Biomedical Engineering\\
	}

\maketitle 
 \hrulefill

\section{Introduction}
 RAID is an acronym for Redundant Array of Independent Disks. With RAID enabled on a storage system you can connect two or more drives in the system so they act as one large volume fast drive or set them up as one system drive used to automatically and instantaneously duplicate (or mirror) your data for real-time backup. \\
 RAID works by placing data on multiple disks and allowing input/output (I/O) operations to overlap in a balanced way, improving performance. Because using multiple disks increases the mean time between failures, storing data redundantly also increases fault tolerance.

RAID arrays appear to the operating system (OS) as a single logical drive.

RAID employs the techniques of disk mirroring or disk striping. Mirroring will copy identical data onto more than one drive. Striping partitions help spread data over multiple disk drives. Each drive's storage space is divided into units ranging from a sector of 512 bytes up to several megabytes. The stripes of all the disks are interleaved and addressed in order. Disk mirroring and disk striping can also be combined in a RAID array.



\end{document}